
\noindent\textbf{Computer vision}: 
%
We extracted the mouse position from each video frame using thresholding,
dilation, contour finding functions in the OpenCV library~\cite{openCV}.

\noindent\textbf{Linear dynamical system model}:
%
We used linear a dynamical system with a six-dimensional state space
$\mathbf{x}(n)=[x(n), \dot{x}(n), \ddot{x}(n), y(n), \dot{y}(n),
\ddot{y}(n)]^\intercal$ (representing the true positions, velocities and
accelerations, along the vertical and horizontal axes)
to model position measurments $\mathbf{y}(n)=[m_x(n), m_y(n)]^\intercal$.
The state equation dynamics followed
the Discrete Wiener Process Accceleration model~\cite{barShalomEtAl01}.

\noindent\textbf{Statistical inference}:
%
Using the Kalman filter and Kalman smoothing algorithm~\cite{kalman61} we
inferred denoised positions, velocities and accelerations (i.e., $\mathbf{x}$)
from noisy position measurements (i.e., $\mathbf{y}(n)$).

\noindent\textbf{Statistical learning}
%
The linear dynamical sytems model contained free parameters (the variance of
the state acceleration discrete Wiener process, the variances of the horizontal
and vertical measurements, and the mean and of the diagonal value of the
covariance of the initial state). We set their values to those that maximized
the likelihood~\cite{shumwayAndStoffer16} of the position measurements.
Maximization was performed using the \texttt{optim} library of
PyTorch\cite{pytorch}.

\noindent\textbf{Simulations}
%
To assess the quality of the previous methods, we sampled states
($\mathbf{x}(n)$) and position measurements ($\mathbf{y}(n)$) from the model in
Section~\ref{sec:ldsModel}. We then evaluated the performance of the previous
methods to recover the sampled states from the sampled positions.
