
We measure the root-means-square error between the true positions, velocities
and accelerations and those estimated by the Kalman filter, Kalman smoother and
a baseline finite differences method. We compared the performance of Kalman
filters and smoothers with manually chosen and learned parameters (maximum
likelihood).

\noindent\textbf{Simulations}
%
Despite large amounts of noise introduced into the simulated measurements, and
the missing observations, both the Kalman filter and smoother yielded accurate
estimates of true positions.
%
The Kalman filter and smoother provided more accurate estimates of velocities
and accelerations than the baseline finite differences method.
%
We also obtained more accurate estimates with learned than with manually set
parameters.
%
Estimates by the Kalman smoother were more accurate than those by the Kalman
filter, specially for missing observations.

\noindent\textbf{Mouse tracking}
%
The interactive figure shows the positions extracted with the computer vision
functions and those inferred by the Kalman filter and smoother. At time with
missing observations (i.e., missing black dots) the Kalman smoother provides
more reasonable position estimates than the Kalman filter, that sometimes goes
astray.

\begin{figure}
    \begin{center}
        \href{http://www.gatsby.ucl.ac.uk/~rapela/fwg/reports/learning/figures/positions_smoothed_session003_start0.00_end15548.27_startPosition0_numPosition10000_pos.html}{\includegraphics[width=3in]{figures/positions_smoothed_session003_start0.00_end15548.27_startPosition0_numPosition10000_pos.png}}
        \caption{Positions measured with the computer vision functions (black) and Kalman filtered (red) and smoothed (green) positions. Click on the figure to access its interactive version.}
    \end{center}
\end{figure}

As with simulated data, estimates of velocities and acceleration by the
baseline finite differences method appeared noisy, and those from the Kalman
filter and smoother seemed less noisy. Extrapolating from the simulation results,
we infer that, also for the mouse tracking, Kalman filter and smoother velocity
and acceleration estimates are more accurate than those of the finite
difference method.
