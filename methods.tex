
\noindent\textbf{Computer vision}: 
%
We extracted the mouse position in each video frame using thresholding,
dilation, contour finding functions in the OpenCV library.

\noindent\textbf{Linear dynamical system (LDS) model}:
%
we used a linear dynamical system with a six-dimensional state space
$\mathbf{x}(n)=[x(n), \dot{x}(n), \ddot{x}(n), y(n), \dot{y}(n),
\ddot{y}(n)]^\intercal$ and mouse position measurements $\mathbf{y}(n)=[m_x(n),
m_y(n)]^\intercal$.  The state dynamics followed the Discrete Wiener
Process Acceleration model~\cite{c1}.

\noindent\textbf{Statistical inference}:
%
using the Kalman filter and smoothing algorithms\cite{c2} we
inferred denoised positions, velocities and accelerations (i.e., $\mathbf{x}(n)$)
from noisy position measurements (i.e., $\mathbf{y}(n)$).

\noindent\textbf{Statistical learning}:
%
the linear dynamical systems model contained free parameters We set their values
to those that maximised the likelihood of the position measurements under the
LDS model~\cite{c2}, using the \texttt{optim} library of PyTorch.

\noindent\textbf{Simulations}:
%
to assess the quality of the previous methods, we sampled states
($\mathbf{x}(n)$) and position measurements ($\mathbf{y}(n)$) from the LDS
model and attempted to recover the states from the measurements.
